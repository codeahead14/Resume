\cvsection{Experience}
\begin{cventries}
  \cventry
    {Scientist/Engineer - SC}
    {Space Applications Center, ISRO}
    {Ahmedabad, India}
    {September 2014 - Present}
    {Involved in the design and development of FPGA-based systems for Microwave Remote Sensing Satellites.  In addition to design and implementation of bare VHDL
architectures, I have been involved in designing data acquisition and processing SoCs making use of the Xilinx-Microblaze Microprocessor and the AXI 4 Bus
Protocol. \linebreak
	\textbf{Onboard SAR Processor} \linebreak
     \begin{cvitems}
        \item {Involved in the hardware design and development of 1st ever 5-FPGA (Virtex 6) board for real time Onboard Synthetic Aperture Radar (SAR) data
        processing.}
        \item {Developed Xilinx Microblaze-based SoC for Real-Time Range-Doppler SAR processor.}
        \item {Developed several AXI 4-compliant IP cores for high speed serial data transfers using the Serializer-Deserializer (SerDes) protocol and LVDS interface for communication between multiple FPGA-based boards as well as for data transfer from external system to onboard DDR3 SDRAM using AXI DMA.}   
        \item {The onboard SAR Processor will be first of its kind in the country and will allow Real Time data processing using complex SAR algorithms providing resolution less than 1 m.}
        \item[]
    \end{cvitems} 
    \textbf{Scatterometer Processor} \linebreak
    \begin{cvitems}
    	\item {Involved in the design of Onboard processor for the successful ScatSAT-1 Mission, launched in September 2016.}
        \item{ Contributed in the design and development of hardware and software systems for High Resolution Scatterometer mission employing several SAR techniques onboard for the first time.}
        \item[]
    \end{cvitems}
    \textbf{Digital Interface Board (DIB)} \linebreak
    \begin{cvitems}
    	\item{Developed, integrated and verified FPGA software for DIB, which is a small form factor pluggable hardware and consists of several high speed interfaces such as SerDes, Gigabit Transceivers (GTX), and LVDS.}
    \end{cvitems}
}
  \cventry
    {Under the guidance of Associate Prof. Dr. B.S. Manoj}
    {Undergraduate Research, Wireless Sensor Networks}
    {Thriuvananthapuram, India}
    {December 2013 - May 2014}
    { Part of the 3-member institute team working in collaboration with University of California, Irvine on Disaster management and information gathering in Shanty town networks. It included study and design of suitable routing protocols, node mobility models and performance evaluation of data routing and information gathering strategies. \linebreak
      \begin{cvitems}
        \item {Studied and researched application of Delay Tolerant Networks (DTNs) to the domain of information gathering in Shanty town network.}
        \item {Studied and evaluated several existing DTN Routing protocols using the ONE (Opportunistic Network Environment) simulator over the Dharavi Shanty town region.}
        \item{ Designed and developed a Python-based DTN simulator with functionalities including the ability to import OpenstreetMap formats for simulation over real-world road networks, user-configurable deployment of Mobile and Stationary sensor nodes, user-configurable Routing protocols and Mobility Models. The simulator is open source and available as  \href{https://github.com/codeahead14/pySim}{github repository}
        }
      \end{cvitems} 
    }
  \cventry
    {Avionics Engineer Trainee}
    {National Aerospace Laboratories (Council of Scientific and Industrial Research)}
    {Bangalore, India}
    {December 2012}
    {Received the opportunity to work on fixed-wing UAVs under the eminent scientists of National Aerospace Laboratories. \linebreak
      \begin{cvitems}
        \item {Studied the dynamics of fixed-wing aircraft and PID controller design for implementing the Auto-pilot mode.}
        \item{Designed a fully-functional MATLAB simulation model for verifying several flight modes such as Roll-auto correction, Pitch-auto correction, Yaw-auto correction and full auto-pilot mode.}
        \item{Implemented and successfully verified and flight-tested the Auto pilot mode on ArduPilot hardware which was integrated with a fixed-wing UAV.}
      \end{cvitems}
    }
  %\begin{cvsubentries}
      %  \cvsubentry{}{KNOX(Solution for Enterprise Mobile Security) Penetration Testing}{Sep. 2013}{}
      %  \cvsubentry{}{Smart TV Penetration Testing}{Mar. 2011 - Oct. 2011}{}
      %\end{cvsubentries}
    
\end{cventries}
